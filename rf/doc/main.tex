\documentclass[titlepage,a4paper]{article}
\usepackage[a4paper,includeheadfoot,margin=2.54cm]{geometry}
\usepackage{tcolorbox}
\usepackage{graphicx}
\usepackage{listings}
\usepackage{glossaries}
\makeglossaries
\usepackage{float}

\graphicspath{{img/}}
\usepackage{hyperref}
\usepackage{cleveref}

\newcommand\imgwidth{0.5\textwidth}
\newcommand\centrefigurestart{\begin{figure}[H]\begin{center}}
\newcommand\centrefigureend{\end{center}\end{figure}}

\newcommand{\cfbox}[2]{\vspace{1cm}%
    \colorlet{currentcolor}{.}%
    {\color{#1}%
    \fbox{\color{currentcolor}{%
    \begin{minipage}{14cm}\begin{center}\vspace{5mm}%
    #2%
	\end{center}\vspace{5mm}\end{minipage}}}\vspace{1cm}%
}%
}

\newacronym{sdr}{SDR}{Software Defined Radio}
\newacronym{RF}{RF}{Radio Frequencies}
\newacronym{mitm}{MITM}{man in the middle}
\newacronym{ide}{IDE}{Integrated Development Environment}
\newacronym{vhf}{VHF}{very high frequency}
\newacronym{uhf}{UHF}{ultra high frequency}
\newacronym{gui}{GUI}{graphical user interface}
\newacronym{lpf}{LPF}{low pass filter}
\newacronym{ascii}{ASCII}{American Standard Code for Information Interchange}
\newacronym{grc}{GRC}{gnuradio companion}
\newacronym{lsb}{LSB}{least significant bit}


\newglossaryentry{repo}{
	name=repository,
	description=a storage location from which software packages may be retrieved and installed on a computer
}

\newglossaryentry{timesink}{
	name=time sink,
	description=a gnuradio block which shows the signal as amplitude against time
}

\newglossaryentry{fftsink}{
	name=FFT sink,
	description=a gnuradio block which shows the signal as amplitude against frequency
}

\newglossaryentry{source}{
	name=source,
	description=a gnuradio block which brings data into the flow
}

\newglossaryentry{samplerate}{
	name=sample rate,
	description=how quickly the analog value is turned into a digital one and passed to the gnuradio flow
}

\newglossaryentry{cutoff}{
	name=cutoff frequency,
	description=that max (or min) frequency a filter will let through
}

\newglossaryentry{transition}{
	name=transition width,
	description=how slopey the edge of a filter are
}

\newglossaryentry{squelch}{
	name=squelch,
	description=a gnuradio block that stops any frequency below a set threshold
}

\newglossaryentry{qd}{
	name=quadrature demod,
	description=a gnuradio block that turns frequency changes into amplitude changes
}

\newglossaryentry{crmm}{
	name=clock recovery MM,
	description=a gnuradio block that synchronises the edge of a binary signal to a clock
}

\newglossaryentry{binslice}{
	name=binary slicer,
	description=a gnuradio block that rounds values to ones and zeros
}

\newglossaryentry{deviation}{
	name=deviation,
	description=how far from the centre frequency a peak it
}

\newglossaryentry{baudrate}{
	name=baud rate,
	description=how many bits are sent per second
}

\newglossaryentry{cac}{
	name=correlate access code - tag,
	description=a gnuradio block that finds a code in the bit stream
}

\newglossaryentry{preamble}{
	name=preamble,
	description="LISTEN TO ME I'M TRANSMITTING"
}

\newglossaryentry{sink}{
	name=sink,
	description=a gnuradio block that lets data leave the flow
}

\begin{document}
\title{RF Workshop}
\author{Patrick Mintram}
\maketitle

\tableofcontents
\listoffigures
\newpage

\section{Introduction}
This guide has been produced to help you work through the RF workshop as part of your work experience. In this workshop you will learn \begin{enumerate}\item What things use RF. \item How we can make something that uses RF. \item How using RF can expose your projects to vulnerabilities. \item What tools we can use to help when using RF. \end{enumerate}.

By the end of this workshop you should be able to use gnuradio, and an SDR to perform a MITM attack on a system which has two devices talking to each other via RF. If you don't know what these things are yet, that's ok, you will find out by working through this guide.

\subsubsection{Using this guide}
There may part of this guide which aren't explained very in depth, that is because the subject of RF and signals is really complicated, so the detail has been left out. If you want to find out more there are some good overviews available online\footnote{http://www.ti.com/lit/ml/slap127/slap127.pdf, for example}. This guide is meant at more of a practical wrokshop than an academic exercise, so if something is glossed over a useful link will be provided in the footnotes, as you have already seen.

\newpage

\section{Equipment}
In order to complete this guide you will need the following equipment.

\begin{enumerate}
\item Laptop with the following: 
\begin{enumerate}
\item The \gls{sdr} Drivers. These are usually available from the manufacturers website.
\item The Arduino IDE\footnote{https://www.arduino.cc/en/Main/Software}.
\item The RadioHead-Extras library should be installed and made available to the Arduino IDE. The library is in the \verb|src| folder of this repo.
\item gnuradio\footnote{https://www.gnuradio.org}.
\item A clone of this repo and performed recursively\footnote{git clone --recursive https://github.com/geekskick/wex-guide}.
\end{enumerate}
\item An SDR with an appropriate antenna for looking at the 430-440MHz frequency range. Its up to you which you use, there are loads available for a reasonable price\footnote{https://www.rtl-sdr.com}.
\item Two Adafruit Feathers with an RFM69 packet radio module attached\footnote{https://learn.adafruit.com/adafruit-feather-m0-radio-with-rfm69-packet-radio/overview} as shown in \cref{adafruit}. These should ideally have antennas attached as described in the Adafruit documentation\footnote{https://learn.adafruit.com/adafruit-feather-m0-radio-with-rfm69-packet-radio/antenna-options}. 

\centrefigurestart
\includegraphics[width=\imgwidth]{feather.jpg}
\caption{An Adafruit Feather M0 with RFM69 Packet Radio}
\label{adafruit}
\centrefigureend

\end{enumerate}t
\newpage

\section{Looking at the Spectrum}
The first thing we need to understand is what the \gls{RF} spectrum looks like. There are plenty of electromagnetic waves around, which you may or may not be aware of. Let's quickly revise what a wave looks like, by looking at \cref{wave}\footnote{\url{https://www.bbc.com/bitesize/guides/zgf97p3/revision/1}}.

\centrefigurestart
\includegraphics[width=\imgwidth]{bitesize_Wave.png}
\caption{The RF spectrum}
\label{wave}
\centrefigureend

The key thing we care about from this diagram is the wavelength because that determines how long it takes for the wave to happen; it's period. This can be used to calculate how many times it' repeats in a second, this is measured in \textit{Hertz} and is a result of the equation shown in \cref{equ:freq}. 

\begin{equation}
\text{Frequency (Hz)} = \frac{1}{\text{Time for one cycle of the wave (s)}}
\label{equ:freq}
\end{equation}

For example a signal that repeats every 2.309469 nanoseconds has a frequency of 434MHz - it repeats 434 million times a second. There are loads of different frequencies in the \gls{RF} spectrum and 434MHz fits in the \gls{uhf} part of this, as shown in \cref{spectrum}\footnote{\url{https://www.ecnmag.com/blog/2017/06/understanding-rf-spectrum}}.

\centrefigurestart
\includegraphics[width=\textwidth]{spectrum_range_services_infographic.png}
\caption{The RF spectrum}
\label{spectrum}
\centrefigureend

We can easily see the effect of these signals by using equipment which uses them; if our radio doesn't work then we know that the signals aren't present at ~90MHz. What about if we want to see a signal at 434MHz though? The radios in our car only tune into parts of the \gls{vhf} frequencies so we can't use those. Here is where our \gls{sdr} comes in useful because we can tell it a frequency to tune into (centre frequency) and we can tell it how quickly to process data (sample rate). We can then plug this into the gnuradio software to see on a graph which frequencies are most powerful, as seen in \cref{uhd_fft}.

\centrefigurestart
\includegraphics[width=\textwidth]{uhd_fft.png}
\caption{The output of the uhd\_fft command with a 434MHz signal present.}
\label{uhd_fft}
\centrefigureend

The spike on the yellow line in \cref{uhd_fft} shows that there is some signal present at the 434MHz frequency. We can change the settings on this window to see other signals which might be present, but first you need to run the following command from the command line to open the window. \verb|uhd_fft| is a provided command which opens a spectrum analyser like the one in \cref{uhd_fft}

\cfbox{red}{ uhd\_fft -f 434000000 -s 10000000 }

Try looking somewhere around 2.4GHz and seeing how busy it is, by changing the rx tune frequency highlighted in \cref{uhd_fft}. Alternatively you can run the command again with different values, the number after the \verb|-f| is the centre frequency, and the number after the \verb|-s| is the sample rate, or how wide the spectrum is.

\subsection{gnuradio flows}
Rather than using a prepackaged command like \verb|uht_fft| we can make our own \gls{gui}s using gnuradio companion. This can be a bit weird, so to start you off one has been provided. From the command line enter:

\cfbox{red}{ gnuradio-companion } 

\cfbox{blue}{ Open up the provided file FFT.grc and click the play button highlighted. } 

You should see the spectrum as we did before. At this point it's worth taking some time to familiarise yourself with gnuradio using the tutorial available here: \url{https://wiki.gnuradio.org/index.php/Guided\_Tutorial\_GRC}, that way it wont be a shock if the instruction is to 'add in the Throttle block', for example.

\centrefigurestart
\includegraphics[width=\textwidth]{grc_fft.png}
\caption{gnuradio flow for viewing the spectrum}
\label{grc_fft}
\centrefigureend
\newpage

\section{Sending a secret message}
In this section we will use our feathers to send some messages to each other. Fortunately this is pretty simple to get started with thanks to the code provided by the RadioHead-Extras library provided.

Open up the \textit{SimpleFSKSend.ino} sketch and the \textit{SimpleFSKRx.ino} files in the arduino editor and load them on to different feathers. You should see that one is sending a message and the other is receiving it, and printing it to the serial monitor. This is our secret message!
\newpage

\section{Spying on a secret message}
In this section we will use our gnuradio flow to perform a \gls{mitm} attack on our two feathers, as shown in \cref{mitm}.

\centrefigurestart
\includegraphics[width=\textwidth]{mitm.png}
\caption{MITM attack on our two feathers}
\label{mitm}
\centrefigureend

Using the gnuradio flow from before, find the frequency which the feather is transmitting on. It's probably going to be the loudest signal you can see, since the transmitter is close to the antenna. 

\subsection{Filtering}

The first step is to use an \gls{lpf} to make sure we receive only the transmission we are interested in. Drag an \gls{lpf} into the gnuradio flow, and add another QT Frequency Sink to see the output. In addition, it's a good idea to add some QT Range Widgets so you can change the parametres of the filter as it's running. Our aim for the \gls{lpf} is to have an output that looks abit like barad-dur.

\begin{figure}
    \centering
    \begin{minipage}{0.45\textwidth}
        \centering
        \includegraphics[width=\textwidth]{barad_dur.jpg} 
        \caption{Sauron's home.}
    \end{minipage}\hfill
    \begin{minipage}{0.45\textwidth}
        \centering
        \includegraphics[width=\textwidth]{fsk_plot.png}
        \caption{FSK Signal as seen on a frequency plot.}
    \end{minipage}
\end{figure}
\newpage

\section{Real World Examples and Further Projects}
\subsection{Tyre Pressure Monitoring System}
This technique can be used in the 'real world' as demonstrated here: \url{https://www.sharebrained.com/downloads/toorcon/dude\_wheres\_my\_car\_toorcon\_sd\_2013.pdf} where someone has used the same technology to find out the tyre pressure of different cars near them. Work through the slides and the code provided and see what you can see.

\subsection{FM Receiver}
You can listen to the radio, and watch the spectrum and signals change as you do it by following the guide here: \url{https://www.instructables.com/id/RTL-SDR-FM-radio-receiver-with-GNU-Radio-Companion/}

\subsection{Satellite Images}
There is a rough guide for receiving images from weather satellites available which may require some special configuration of antenna, if you're up for it! 

\url{http://oz9aec.net/radios/gnu-radio/noaa-weather-satellite-reception-with-gnu-radio-and-usrp/}

\newpage
\printglossary
\end{document}
