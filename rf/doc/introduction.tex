This guide has been produced to help you work through the \gls{RF} workshop as part of your work experience. In this workshop you will learn \begin{enumerate}\item What things use \gls{RF}. \item How we can make something that uses \gls{RF}. \item How using \gls{RF} can expose your projects to vulnerabilities. \item What tools we can use to help when using \gls{RF}.\end{enumerate}

\subsection{What you will be doing}
In following this workshop you will be using tools available at home, to look at some of the information being sent through the air as \gls{RF}. You will be able to see the different frequencies used by different kinds of devices, such as doorbells, WiFi, remote control cars and bluetooth connected items. You will then send some secret messages between some microcontrollers and use these tools to spy on the message as part of a \gls{mitm} attack.

\subsection{Using this guide}
There may part of this guide which aren't explained very in depth, that is because the subject of \gls{RF} and signals is really complicated, so the detail has been left out. If you want to find out more there are some good overviews available online\footnote{http://www.ti.com/lit/ml/slap127/slap127.pdf, for example}. This guide is meant at more of a practical workshop than an academic exercise, so if something is glossed over a useful link will be provided in the footnotes, as you have already seen. It isn't expected that you full understand the subjects covered, but it is expected that you'll take some time in the future to have a play with the tools and techniques and learn a bit more about the things you've touched on.

\subsection{Feedback}
The author of this guide is keen to know what you think; the good, the bad and the ugly. Please feel free to send any comments their way, or if you're that way inclined use the github system and raising an issue or creating a pull request. 
