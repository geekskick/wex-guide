In this section we will make a custom gnuradio flow to perform a \gls{mitm} attack on our two feathers, as shown in \cref{mitm}.

\centrefigurestart
\includegraphics[width=\textwidth]{mitm.png}
\caption{MITM attack on our two feathers}
\label{mitm}
\centrefigureend

In this section will will walk through the steps to looking at the data send over the air.

\centrefigurestart
\includegraphics[width=\textwidth]{gnuflow_flow}
\caption{The steps to looking at the data sent}
\label{mitm}
\centrefigureend


\subsection{Connecting the Radio}
In order to get data into the flow from the \gls{sdr} we must use a source. In this case, it's the RTL-SDR source (In my examples I will be using a slightly different one, however the main points are the same) if there are issues there are plenty of resources available online to work through the RTL-SDR specifics \footnote{\url{https://www.instructables.com/id/RTL-SDR-FM-radio-receiver-with-GNU-Radio-Companion/}}.

Drag in the RTL-SDR source to the canvas.
\centrefigurestart
\includegraphics[width=\textwidth]{rtl_src.png}
\caption{A gnuradio flow with just a radio source}
\centrefigureend

If you double click the block you will get some options, in here the key things to find are the \textit{sample rate}, \textit{gain}, and \textit{centre frequency}. The sample rate affects how quickly the radio is taking measurements (the rate at which it samples!) and as a result how wide the spectrum we are monitoring is. The gain is how loud that measurement is. The centre frequency is where the middle of the spectrum we are observing sits. The best way for us to experiment with these settings is to use the \textit{QT GUI Range} widget, this widget givs us a slider on the GUI to change settings at runtime. In order to do this, first, double click the widget's block and give it a meaningful ID, something like \textit{gain\_slider}. Give it a start value of 0, a stop value of 1, and a step of 0.1.

\centrefigurestart
\includegraphics[width=0.5\textwidth]{range_settings}
\caption{Setting up a slider}
\centrefigureend

Now you can enter it's unique ID in the radio's settings under \textit{gain}. This is effectively telling the radio to get it's gain value from the slider called \textit{gain\_slider}.

\centrefigurestart
\includegraphics[width=0.5\textwidth]{gain_slider_in_usrp}
\caption{Using a slider value for the gain in the radio source.}
\centrefigureend

Do the same, creating a slider for the \textit{centre frequency} and \textit{sample rate} with the following settings:

\begin{table}[H]
\begin{tabular}{|c|c|c|c|c|}
\hline
Setting & Start & Stop & Default & Step \\ \hline
Centre Frequency & 430000000 & 440000000 & 435000000 & 500000 \\ \hline
Sample Rate & 1000000 & 10000000 & 2000000 & 500000 \\ \hline
\end{tabular}
\caption{Radio slider settings}
\end{table}

\subsubsection{Viewing the spectrum}
The next thing is to actually be able to see the spectrum! For this a \textit{QT GUI Frequency Sink} is required. Drag this in and connect it to the radio by clicking the blue tabs.

\centrefigurestart
\includegraphics[width=\textwidth]{basic}
\caption{A basic flow.}
\centrefigureend

Again, open the frequecy sink block and set the centre frequency to the same slider as your radio. Set the bandwidth to your sample rate slider:

\centrefigurestart
\includegraphics[width=0.5\textwidth]{fft_sink}
\caption{FFT Sink Configuration.}
\centrefigureend

While you're there, go to the \textit{Config} tab and select \textit{yes} to control panel. Now run your flow by clicking the play button at the top of the gnuradio window. This will turn your flow into a python script and run it. If there are errors address these before continuing. All being well, the window that pops up should have three sliders, and a very familiar looking spectrum. Take some time to play with the sliders and see what they do to the spectrum view. (It helps if you enable \textit{max hold} in the spectrum control panel). 

\subsection{Isolating the good bits}

The next step is to use a \gls{lpf} to make sure we receive only the parts of the transmission we are interested in. Drag an \gls{lpf} into the gnuradio flow, and add another QT Frequency Sink to see the output. In addition, it's a good idea to add some more QT Range Widgets so you can change the parameters of the filter as it's running. The parameters of interest here are the \textit{cutoff frequency} (how wide the filter is) and the \textit{transition width} (how slopey it's edges are). 

\centrefigurestart
\includegraphics[width=0.5\textwidth]{lpf_filter}
\caption{Where the cutoff frequency and transition width matter.}
\centrefigureend

Our aim for the \gls{lpf} is to have an output that looks abit like barad-dur.

\begin{figure}[H]
    \centering
    \begin{minipage}{0.45\textwidth}
        \centering
        \includegraphics[width=\textwidth]{barad_dur.jpg} 
        \caption{Sauron's home.}
    \end{minipage}\hfill
    \begin{minipage}{0.45\textwidth}
        \centering
        \includegraphics[width=\textwidth]{fsk_plot.png}
        \caption{FSK Signal as seen on a frequency plot.}
    \end{minipage}
\end{figure}

Run your new flow and you should be able to use the filter sliders to isolate specific parts of your signal. I found a value of around 20000 for both to give pretty good results during experimenation. We can further help our filter by using \textit{squelch}. This block will only let signals above a certain amplitude through, effectively cutting out all the noise. Put this block after your filter, and set it's \textit{alpha} to 1. Again, make a slider for it's squelch value and play with it - experiments proved that a value of -40 gave good results for me.

\centrefigurestart
\includegraphics[width=0.5\textwidth]{lpf_peaks}
\caption{An ideal output from the LPF}
\centrefigureend


\centrefigurestart
\includegraphics[width=\textwidth]{squelch}
\caption{Radio - LPF - Squelch - Frequency Sink}
\centrefigureend

\subsection{Get signal in terms of time}
Now we have isolated the peaks, we can turn these into bits by using a \textit{Quadradure Demod} block with a gain setting of 1. Plug the output from this into a \textit{QT GUI Time Sink}, and enable the time sink's control panel the saem way you did with the frequency sink. 

\centrefigurestart
\includegraphics[width=\textwidth]{dq_flow}
\caption{Radio - LPF - Squelch - Quad Demod - Time Sink}
\centrefigureend

This new graph will have turned the frequency changes into changes against time. This will be easier to see in the Time Sink if you set the trigger to auto, and when the flow is running click '+ X Max' loads until you can see what looks like something interesting.


\centrefigurestart
\includegraphics[width=0.5\textwidth]{trig_auto}
\caption{Setting the timesink trigger to auto in the GUI}
\centrefigureend



\centrefigurestart
\includegraphics[width=\textwidth]{qd_time}
\caption{Our signal against time}
\centrefigureend

\subsection{Turn into binary}

In order to turn the signal into binary we need to know exactly when the transition happens. In this section we will do that using the \textit{Clock Recovery MM} block and a \textit{binary slicer}. The clock recovery relies on knowing the deviation (how far apart your peaks are in the frequency plot) and your baud rate (how quickly bits are sent in the data). Since we made the feathers talk to each other, we know these settings - so it becomes a reltively easy task of plopping numbers into an equation. In the real world however the process of finding this information may take some time and extra snooping around the RF spectrum and settings. Drag in 3 \textit{variable} blocks and use the following information:

\begin{table}[H]
\begin{tabular}{|c|c|}
\hline
Variable Name & Value \\ \hline
baudrate\_hz & 9600 \\ \hline
deviation\_hz & 19200 \\ \hline
samples\_per\_symbol & sample\_rate/baudrate\_hz \\ \hline
\end{tabular}
\end{table}

Then, in your clock recovery block set the \textit{omega} to be \textit{samples\_per\_symbol}.

\centrefigurestart
\includegraphics[width=0.5\textwidth]{clock_recovery}
\caption{Clock recovery block settings}
\centrefigureend

Again, connect a time sink to this to see the output.


\centrefigurestart
\includegraphics[width=\textwidth]{clk_recovery_flow}
\caption{Radio - Filter - Squelch - Quad Demod - Clock Recovery}
\centrefigureend

What you should see is a slight difference between the time sinks for just the quad demod and the clock recovery. The clock recovery one will look a bit pointier, and the lines will be less thick. This is clearest at the start of the graph - one data point will be high, and the next low whereas without the clock recovery this part of this signal has 'fatter' highs and lows.

\centrefigurestart
\includegraphics[width=\textwidth]{clock_vs_qd_time}
\caption{The signal before and after clock recovery}
\centrefigureend

\centrefigurestart
\includegraphics[width=\textwidth]{clk_recovery_vs_qd_zoom}
\caption{Detail of a series of 1 and 0 before and after clock recovery}
\centrefigureend

Finally add a \textit{binary slicer} to the end of the clock recovery block. Done.
\subsection{Find start}
Finding the start of the signal is relatively easy as a block called \textit{Correlate Access Code - Tag} will do this by searching for a pattern in the binary stream coming in. Through inspection of the time sinks before we can see that the signal starts with a long stream of 1 and 0. This is known as the preamble and is present in lots of signals; it's the radio's way of saying 'LISTEN TO ME I'M TRANSMITTING' before it sends any useful data. In your new block set the \textit{Access Code} to be 1010101010101010101010101010101000101101 (or 4 bytes of 10101010 then a byte of 00101101), a threshhold of 0, and call the tag 'preamble'. 

\centrefigurestart
\includegraphics[width=0.5\textwidth]{cac_settings}
\caption{Correlate Access Code Settings}
\centrefigureend

This block, when it sees a stream of data matching the access code will insert a special tag into the data stream with the name 'preamble' to show that it's detected a preamble. Finally output this to a file using a \textit{File Meta Sink}. This file may grow quite large so I recommend making something like "/tmp/file\_sink" so that it's deleted when you turn off your computer.

\centrefigurestart
\includegraphics[width=\textwidth]{full_Flow}
\caption{Radio - LPF - Squelch - Quad Demod - Clock Recovery - Binary Slicer - Correlate Access Code - File Sink}
\centrefigureend
\subsection{Decode}
This is where it can get abit tougher! I have cobbled together a small tool to help. It's not that great and misses some of the messages, however it's a start! In the repo you downloaded use the analyser to examine the contents of the file you created. This tool looks for the 'preamble' tag in the file and then prints off a certain number of bytes afterwards to the screen. In order to build the tool navigate to it's directory and enter the command 'source doit.sh'. This will creat a folder called 'build' and make the application there. Then you can use it by entering the command './analyser /tmp/file\_sink' and it should print off some of the packets.

\centrefigurestart
\includegraphics[width=\textwidth]{packets}
\caption{Output of the analyser}
\centrefigureend

It appears to skip every other packet at this stage so, however we can clearly see a bunch of 'Hello World' messages decoded - congratulations you have successfully performed a \gls{mitm} attack on your feather!

A full flow is available to do this in the repo - if you get stuck at any point.
